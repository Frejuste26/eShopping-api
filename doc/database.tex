L’architecture de la base de données pour une application e-commerce bien conçue doit être **relationnelle** (par exemple MySQL, PostgreSQL) ou **NoSQL** (MongoDB, DynamoDB), selon les besoins spécifiques. Une structure relationnelle est souvent préférée pour sa robustesse et sa capacité à gérer des relations complexes.

Voici une **modélisation relationnelle** typique pour un site e-commerce.

---

### **1. Tables principales**

#### **1.1. Utilisateurs (Users)**
- `id` (PK) : Identifiant unique.
- `email` : Adresse email unique.
- `password` : Mot de passe crypté.
- `name` : Nom de l'utilisateur.
- `phone` : Numéro de téléphone.
- `address` : Adresse par défaut (peut être une relation si plusieurs adresses).
- `role` : Rôle (client, admin, etc.).
- `created_at` : Date de création.
- `updated_at` : Date de dernière modification.

---

#### **1.2. Produits (Products)**
- `id` (PK) : Identifiant unique.
- `name` : Nom du produit.
- `description` : Description détaillée.
- `price` : Prix unitaire.
- `stock` : Quantité en stock.
- `category_id` (FK) : Référence à la catégorie.
- `brand_id` (FK) : Référence à la marque.
- `created_at` : Date de création.
- `updated_at` : Date de dernière modification.

---

#### **1.3. Catégories (Categories)**
- `id` (PK) : Identifiant unique.
- `name` : Nom de la catégorie.
- `parent_id` (FK, nullable) : Référence pour les sous-catégories (auto-relation).
- `created_at` : Date de création.
- `updated_at` : Date de dernière modification.

---

#### **1.4. Marques (Brands)**
- `id` (PK) : Identifiant unique.
- `name` : Nom de la marque.
- `created_at` : Date de création.
- `updated_at` : Date de dernière modification.

---

#### **1.5. Commandes (Orders)**
- `id` (PK) : Identifiant unique.
- `user_id` (FK) : Référence à l'utilisateur.
- `total_price` : Prix total de la commande.
- `status` : Statut (en attente, payé, expédié, annulé, etc.).
- `created_at` : Date de création.
- `updated_at` : Date de dernière modification.

---

#### **1.6. Détails de commande (Order_Items)**
- `id` (PK) : Identifiant unique.
- `order_id` (FK) : Référence à la commande.
- `product_id` (FK) : Référence au produit.
- `quantity` : Quantité commandée.
- `price` : Prix unitaire au moment de la commande.
- `created_at` : Date de création.
- `updated_at` : Date de dernière modification.

---

#### **1.7. Avis produits (Reviews)**
- `id` (PK) : Identifiant unique.
- `user_id` (FK) : Référence à l'utilisateur.
- `product_id` (FK) : Référence au produit.
- `rating` : Note donnée (1 à 5).
- `comment` : Commentaire textuel.
- `created_at` : Date de création.
- `updated_at` : Date de dernière modification.

---

### **2. Tables auxiliaires**

#### **2.1. Adresses utilisateur (Addresses)**
- `id` (PK) : Identifiant unique.
- `user_id` (FK) : Référence à l'utilisateur.
- `address_line_1` : Première ligne d'adresse.
- `address_line_2` : Deuxième ligne (facultative).
- `city` : Ville.
- `postal_code` : Code postal.
- `country` : Pays.
- `created_at` : Date de création.
- `updated_at` : Date de dernière modification.

---

#### **2.2. Codes promotionnels (Coupons)**
- `id` (PK) : Identifiant unique.
- `code` : Code promotionnel.
- `discount_type` : Type de réduction (pourcentage ou montant fixe).
- `discount_value` : Valeur de la réduction.
- `expiry_date` : Date d'expiration.
- `minimum_order_value` : Montant minimum pour appliquer le coupon.
- `created_at` : Date de création.
- `updated_at` : Date de dernière modification.

---

#### **2.3. Paiements (Payments)**
- `id` (PK) : Identifiant unique.
- `order_id` (FK) : Référence à la commande.
- `payment_method` : Méthode de paiement (carte, PayPal, etc.).
- `transaction_id` : Identifiant de la transaction (généré par le service de paiement).
- `amount` : Montant payé.
- `status` : Statut (réussi, échec, remboursé).
- `created_at` : Date de création.

---

#### **2.4. Images des produits (Product_Images)**
- `id` (PK) : Identifiant unique.
- `product_id` (FK) : Référence au produit.
- `url` : URL de l’image.
- `is_primary` : Indique si c’est l’image principale.
- `created_at` : Date de création.

---

### **3. Relations clés**
1. **Users ↔ Orders** : Un utilisateur peut avoir plusieurs commandes.  
2. **Orders ↔ Order_Items** : Une commande contient plusieurs produits.  
3. **Products ↔ Categories** : Un produit appartient à une catégorie.  
4. **Products ↔ Brands** : Un produit est associé à une marque.  
5. **Products ↔ Reviews** : Un produit peut avoir plusieurs avis d'utilisateurs.  
6. **Users ↔ Reviews** : Un utilisateur peut laisser plusieurs avis.  
7. **Addresses ↔ Users** : Un utilisateur peut avoir plusieurs adresses.  

---

Cette architecture est évolutive et peut être adaptée aux besoins spécifiques (par exemple, ajout de gestion de stock avancée, abonnements, ou intégration avec des marketplaces).